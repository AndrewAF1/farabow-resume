\documentclass{article}
\usepackage{geometry}
\geometry{margin=1in}
\usepackage{multicol}
\setlength{\columnsep}{1cm}

\begin{document}
  \begin{center}
    \begin{tabular}{r l}
      {\huge\textbf{Andrew}} & {\huge\textbf{Farabow}} \\
      \hspace{35pt} github.com/andrewaf1 & linkedin.com/in/andrew-farabow \\
      703-474-6270 & aafarabow@gmail.com \\
    \end{tabular}


  \begin{flushleft}
    \begin{multicols}{2}
      {\large\textbf{\underline{\smash{Education}}}} \\
       \textbf{Virginia Tech	(2019 - present)} \\
      Major: General Engineering (Computer Science) \\
      \textbf{Gonzaga College High School	(2015 - 2019)} \\
      GPA 3.98 \\
     

    \columnbreak
    {\large\textbf{\underline{\smash{Skills}}}} \\
    {\textbf{Programming:}} Python, Java, C/C++, Go \\
    {\textbf{Frameworks:}} Numpy, PyTorch, OpenAI Gym, OpenCV, Pandas \\
    {\textbf{Other Skills:}} deep neural networks, deep reinforcement learning, Git, LaTeX \\

    \end{multicols}

    {\large\textbf{\underline{\smash{Work Experience}}}} \\
    \textbf{Decipher Technology Studios Internship \hfill Summer 2018 \& 2019}
    \begin{itemize}
      \item Working on a small team to develop a new product which provides deep reinforcement learning-powered predictive auto-scaling for Decipher's Grey Matter service mesh
      \item Studying various deep reinforcement learning architectures, including Deep Q-Learning, Policy Gradient, Advantage Actor Critic, Proximal Policy Optimization (PPO) and Soft Actor Critic (SAC)
      \item Collaborated with another intern to implement PPO and write a microservice environment simulator
      \item Configured and deployed a demo of Sense to AWS under an imminent deadline
      \item Added Gated Recurrent Units to the network to better leverage time series data
      \item Tweaked hyperparameters and restructured PPO to improve performance
    \end{itemize}


    {\large\textbf{\underline{\smash{Activities}}}} \\
    \textbf{Gonzaga Dramatic Association Stage Crew \hfill 2017 - 2019}
    \begin{itemize}
      \item Led a 20-member team for two productions as stage manager (2018-2019)
      \item Designed and coordinated the construction of a structure over 20 ft. wide and 8 ft. tall
      \item Called cues during shows and solved problems in a high-pressure environment
      \item Maintained safe working conditions for the crew
      \item Worked with the stage manager to quickly diagnose and fix technical issues as assistant stage manager (2017-2018) before being promoted
    \end{itemize}

    \textbf{HackBI  (Bishop Ireton High School Hackathon)}

    \begin{itemize}
      \item Member of a team that won best overall in a programming contest by writing an app that makes use of machine learning and computer vision techniques to interpret hand-written text
      \item Returned to HackBI in 2018 to mentor teams and teach deep learning concepts
    \end{itemize}

    {\large\textbf{\underline{\smash{Projects}}}} \\
    \textbf{Machine Learning Templates} - flexible PyTorch implementations of a neural network, autoencoder and evolutionary algorithm designed for future machine learning projects \\
    \textbf{Grease Lights} - custom Arduino-driven circuit and software to control LED strips located on the set of a high school production of Grease \\
    \textbf{Magic Mirror} - remotely controlled Raspberry Pi-powered theatrical optical illusion



  \end{flushleft}
  \end{center}


\end{document}
