\documentclass{article}
\usepackage{geometry}
\geometry{margin=0.9in}
\usepackage{multicol}
\setlength{\columnsep}{1cm}
\thispagestyle{empty}

\begin{document}
  \begin{center}
    \begin{tabular}{r l}
      {\huge\textbf{Andrew}} & {\huge\textbf{Farabow}} \\
      \hspace{35pt} github.com/andrewaf1 & linkedin.com/in/andrew-farabow \\
      703-474-6270 & contact@andrewfarabow.com \\
    \end{tabular}


  \begin{flushleft}
    \begin{multicols}{2}
      {\large\textbf{\underline{\smash{Education}}}} \\
      \textbf{Virginia Tech (expected grad 2023)} \\
      GPA: 3.25 \quad B.S. in Computer Science \\
      Relevant Courses: Data Structures, Intro to Restricted Research, Discrete Math, Calculus 1-2 \\
      \textbf{Gonzaga College High School	(2015 - 2019)} \\
      GPA 3.98 \\
     

    \columnbreak
    {\large\textbf{\underline{\smash{Skills}}}} \\
    {\textbf{Programming:}} Python, Java, Matlab \\
    {\textbf{Frameworks:}} Numpy, PyTorch, OpenAI Gym, OpenCV, Pandas, Matplotlib, Visdom \\
    {\textbf{Other:}} neural networks (fully-connected, recurrent, and convolutional), reinforcement learning, GANs, autoencoders, data analytics, Linux, Git, Kubernetes, LaTeX, Agile \\

    \end{multicols}

    {\large\textbf{\underline{\smash{Work Experience}}}} \\
    \textbf{Undergraduate Research Assistant - Virginia Tech \hfill 2019 - present}
    \begin{itemize}
      \itemsep0em
      \item Working for the Hume Center for National Security and Technology under Prof. Daniel Doyle to create and train object-detecting convolutional neural networks for drone navigation
      \item Working for the Center for Bioinspired Science and Technology under Prof. Rolf Mueller to interpret sonar data with deep learning and track bats in a lab setting with DeepLabCut
    \end{itemize}


    \textbf{Machine Learning Engineer Intern - Decipher Technology Studios \hfill 2018 - present}
    \begin{itemize}
      \itemsep0em
      \item Working on a small team to develop Sense, a new product which provides deep reinforcement learning-powered predictive autoscaling for Decipher’s Grey Matter service mesh
      \item Studied and implemented policy gradient, Q-Learning, and actor-critic approaches to deep reinforcement learning (DQN, DDPG, A2C, PPO, SAC, etc)
      \item Wrote a microservice environment simulator for offline training with another intern and created a rule-based autoscaler to jumpstart training via imitation learning.
      \item Added recurrent and convolutional layers to the neural networks to better leverage time-series data
      \item Collected metrics using Prometheus and Gatling and tested various model architectures on the data
      \item Created infrastructure to deploy Sense as a service on Openshift and Elastic Kubernetes Service.
    \end{itemize}


    {\large\textbf{\underline{\smash{Activities}}}} \\

    \textbf{IC CAE Associate - The Hume Center for National Security and Technology \hfill 2020 - present}
    \begin{itemize}
      \itemsep0em
      \item Attend talks and workships offered by the Hume Center's National Security Education Program 
    \end{itemize}
    
    \textbf{Judging Coordinator - VTHacks Organizing Team \hfill 2019 - present}
    \begin{itemize}
      \itemsep0em
      \item Reached out to potential corporate sponsors and faculty judges for Virginia Tech's hackathon
      \item Handled judging logistics during the event and took note of improvements to implement next year
    \end{itemize}
    
    \textbf{Stage Manager - Gonzaga Dramatic Association Stage Crew \hfill 2017 - 2019}
    \begin{itemize}
      \itemsep0em
      \item Led a team of over 20 students in the construction of a structure over 20 ft. wide and 8 ft. tall
      \item Called cues during shows, maintained safe working conditions and quickly diagnosed and fixed technical issues in a high-pressure environment
    \end{itemize}

    \textbf{Participant and Mentor - HackBI  (Bishop Ireton High School Hackathon)}
    \begin{itemize}
      \itemsep0em
      \item Won best overall in a programming contest by writing an app that makes use of machine learning and computer vision techniques to interpret hand-written text
      \item Returned to HackBI in 2018 to mentor teams and teach deep learning concepts
    \end{itemize}


    {\large\textbf{\underline{\smash{Projects}}}} \\
    \textbf{Computable AI} - co-author of a blog on machine learning, writing a Fundamentals of Deep RL series \\
    \textbf{Machine Learning Templates} - flexible PyTorch implementations of a supervised learning neural network, autoencoder, GAN, and evolutionary algorithm designed for future machine learning projects \\
    \textbf{Grease Lights and Magic Mirror} - coded and designed circuits for custom Arduino and Raspberry Pi-based lighting effects and optical illusions featured in high school theater productions

  \end{flushleft}
  \end{center}


\end{document}
