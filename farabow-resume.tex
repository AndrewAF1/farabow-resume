\documentclass{article}
\usepackage{geometry}
\geometry{margin=0.5in}
\usepackage{multicol}
\setlength{\columnsep}{1cm}
\thispagestyle{empty}

\begin{document}
  \begin{center}
    \begin{tabular}{r l}
      {\huge\textbf{Andrew}} & {\huge\textbf{Farabow}} \\
      \hspace{35pt} github.com/andrewaf1 & linkedin.com/in/andrew-farabow \\
      703-474-6270 & contact@andrewfarabow.com \\
    \end{tabular}


  \begin{flushleft}
    \begin{multicols}{2}
      {\large\textbf{\underline{\smash{Education}}}} \\
      \textbf{Virginia Tech (expected grad 2023)} \\
      GPA: 3.24 \; B.S. in Computer Science w/ Stat minor \\
      Elective Courses: Restricted Research, Mathematical Statistics 1-2 (Probability and Inference), Intro to Data Analytics and Visualization\\
      \textbf{Gonzaga College High School	(2015 - 2019)} \\
      GPA 3.98 \\
     

    \columnbreak
    {\large\textbf{\underline{\smash{Skills}}}} \\
    {\textbf{Programming:}} Python, C, Java, R, Matlab \\
    {\textbf{Frameworks:}} PyTorch, Scikit-learn, Keras, Numpy, OpenCV, Pandas, Matplotlib, RLLib, OpenAI Gym \\
    {\textbf{Other:}} deep learning, recurrent and convolutional neural networks, reinforcement learning, GANs, autoencoders, data analytics, statistical learning, Linux, Git, Kubernetes, LaTeX, Agile \\

    \end{multicols}

    {\large\textbf{\underline{\smash{Work Experience}}}} \\

    \textbf{Research Assistant - Sanghani Center (Virginia Tech) \hfill May. 2021 - present}
    \begin{itemize}
      \itemsep0em
      \item Working with Prof. Naren Ramakrishnan's group to create an open-source library of epidemiological models for forecasting the COVID-19 pandemic and the seasonal flu.. 
    \end{itemize}
    
    \textbf{Research Assistant - Hume Center (Virginia Tech) \hfill Nov. 2019 - present (school year)}
    \begin{itemize}
      \itemsep0em
      \item Working under Prof. Daniel Doyle to train reinforcement learning agents on strategy games, with the goal of developing new mechanisms to improve training efficiency on very complex tasks. 
      \item Previously designed and trained object-detecting convolutional neural network architectures, which achieved 97\% accuracy on the classification phase of the Lockheed Martin AlphaPilot Dataset and were deployed to a drone's computer to aid in navigation
    \end{itemize}

    \textbf{Research Assistant - BIST (Virginia Tech) \hfill Sept. 2019 - present (school year)}
    \begin{itemize}
      \itemsep0em
      \item Working for the Center for Bioinspired Science and Technology under Prof. Rolf Mueller to use deep learning to facilitate robot navigation via biosonar
      %\item Previously used the DeepLabCut software package to track bats' movements in images. 
      \item Using convolutional neural networks to classify spectrograms based on their location inside the region we collected data from, with the goal of eventually creating an algorithm for navigating through the forest 
    \end{itemize}


    \textbf{Machine Learning Engineer Intern - Decipher Technology Studios \hfill 2018 - 2020 (summers)}
    \begin{itemize}
      \itemsep0em
      \item Worked on a small team to develop Sense, a new product which provides deep reinforcement learning-powered predictive autoscaling for Decipher’s service mesh platform
      \item Implemented a library of policy gradient, Q-Learning, and actor-critic approaches to deep reinforcement learning (DQN, DDPG, A2C, PPO, SAC, etc), as well as model improvements such as recurrent and convolutional layers, with PyTorch
      \item Wrote a microservice environment simulator for offline training with another intern and created a rule-based autoscaler to jumpstart training via imitation learning.
      %\item Added recurrent and convolutional layers to the neural networks to better leverage time-series data
      %\item Collected metrics using Prometheus and Gatling and tested various model architectures on the data
      \item Created infrastructure to deploy Sense as a service on Openshift and Elastic Kubernetes Service.
    \end{itemize}


    {\large\textbf{\underline{\smash{Activities}}}} \\

    \textbf{Judging Coordinator - VTHacks Organizing Team \hfill 2019 - present}
    \begin{itemize}
      \itemsep0em
      %\item Reached out to potential corporate sponsors and faculty judges for Virginia Tech's hackathon
      %\item Handled judging logistics during the event and took note of improvements to implement next year
      \item Responsible for recruiting faculty judges and managing judging logistics during the event.
    \end{itemize}
    
    \textbf{Stage Manager - Gonzaga Dramatic Association Stage Crew \hfill 2017 - 2019}
    \begin{itemize}
      \itemsep0em
      %\item Led a team of over 20 students in the construction of a structure over 20 ft. wide and 8 ft. tall
      %\item Called cues during shows, maintained safe working conditions and quickly diagnosed and fixed technical issues in a high-pressure environment
      \item Led a team of over 20 students in the construction of a structure over 20 ft. wide and 8 ft. tall
      \item Quickly diagnosed and fixed technical issues in a high-pressure environment
    \end{itemize}

    \textbf{Participant and Mentor - HackBI  (Bishop Ireton High School Hackathon) \hfill 2017 \& 2018}
    \begin{itemize}
      \itemsep0em
      \item Won best overall in a programming contest by writing an app that makes use of machine learning and computer vision techniques to interpret hand-written text
      \item Returned to HackBI in 2018 to mentor teams and teach deep learning concepts
    \end{itemize}


    {\large\textbf{\underline{\smash{Projects}}}} \\
    \textbf{Computable AI} - co-author of a blog on machine learning, writing a Fundamentals of Deep RL series \\
    \textbf{Machine Learning Templates} - flexible PyTorch implementations of a supervised learning neural network, autoencoder, GAN, and evolutionary algorithm designed for future machine learning projects \\
    \textbf{Grease Lights and Magic Mirror} - coded and designed circuits for custom Arduino and Raspberry Pi-based lighting effects and optical illusions featured in high school theater productions

  \end{flushleft}
  \end{center}


\end{document}
